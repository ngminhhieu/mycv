% YAAC Another Awesome CV LaTeX Template
%
% This template has been downloaded from:
% https://github.com/darwiin/yaac-another-awesome-cv
%
% Author:
% Christophe Roger
%
% Template license:
% CC BY-SA 4.0 (https://creativecommons.org/licenses/by-sa/4.0/)

\sectionTitle{Academic Achievements}{\faTasks}

%\begingroup
%\setlength{\columnsep}{5pt}%
%\begin{figure}[!h]
\begin{wrapfigure}{r}{0.5\textwidth}
\centering
\includegraphics[width=0.5\textwidth]{figures/rumour.png}
\vspace{-.5cm}
\caption{Misinformation Diffusion}
\vspace{-.5cm}
\end{wrapfigure}
%\end{figure}
%\endgroup

\textbf{1. Advances the field of emergent trust management by developing models for early detection, minimal-effort validation, and efficient visualisation of misinformation on social media platforms:} ICDE'15 (CORE $A^*$, 94 citations), IJCAI'17 (CORE $A^*$, 73 citations) TKDE'18 (IF 9.235, CORE $A^*$, 74 citations), Information Fusion (IF 18.6, Top 3\%), two VLDB'19 (CORE $A^*$, 128 citations), SIGIR'20 (CORE $A^*$), ICDE'22 (CORE $A^*$), VLDBJ'22 (CORE $A^*$, Q1), CVIU'22 (548 citations), CSUR'24 (CORE $A^*$), etc.

Our modern society is struggling with an unprecedented amount of online misinformation, which does harm to democracy, economics, and cybersecurity. Journalism and politics have been impacted by misinformation on a global scale, with weakened public trust in governments seen during the Brexit referendum and viral fake election stories outperforming genuine news on social media during U.S. presidential election campaigns. Online misinformation also single-handedly caused \$136.5 billion in losses in the stock market value through a single fake news about explosions in the White House.

Our works make significant contributions to the field by developing comprehensive models and methodologies for early detection, validation, and visualisation of misinformation on social media platforms.  These advancements address key challenges in the timely and accurate debunking of misinformation, enhancing the effectiveness of misinformation management frameworks and providing valuable insights for future research and practical applications. Our works contribute to economic stability by preventing financial disruptions caused by misinformation, bolsters cybersecurity by addressing AI-driven threats, and safeguards public health by combating false health information. Additionally, the research provides valuable educational tools for promoting digital literacy and critical thinking, and aids media organizations in ensuring the accuracy and credibility of their reporting, thereby improving the quality of information available to the public.


%\begingroup
%\setlength{\columnsep}{5pt}%
%\begin{figure}[!h]
\begin{wrapfigure}{r}{0.5\textwidth}
\centering
\includegraphics[width=0.5\textwidth]{figures/explanation.png}
\vspace{-.5cm}
\caption{Multi-dimension XAI}
\vspace{-.5cm}
\end{wrapfigure}
%\end{figure}
%\endgroup


\textbf{2. Develops user-centric systems, explainable AI techniques, and human-in-the-loop to enhance algorithmic accuracy to bridge the trust gap between humans and AI:} KBS'24 (IF 8.8), WSDM'23 (CORE $A^*$), JBHI'23 (IF 7.7, CORE $A^*$, Top 10\%), Information Sciences'23 (IF 8.233), ESWA'22 (IF 8.665), SIGMOD'21 (CORE $A^*$), Information Systems'19 (CORE $A^*$), ICDE'18 (CORE $A^*$) VLDBJ'17 (CORE $A^*$), WISE'13 (265 citations), etc.

Working with user-centric systems has been a long-time passion. For example, one of my publications revolves around providing example-based explanations for streaming fraud detection on graphs, where we designed techniques to offer the best contextual explanations to get optimal user feedback without bias. I also worked on model-agnostic and diverse explanations for streaming rumour graphs, focusing on visualising these explanations and supporting information effectively. I was the first to propose and analyse the novel problem of leveraging human experts to improve the quality of algorithmic results, combining scarce expert knowledge with redundant crowd knowledge to resolve the trade-off between quality and scalability, an achievement that traditional approaches have struggled to match. Additionally, I developed techniques for example-based explanations with adversarial attacks for respiratory sound analysis, and real-time wildfire detection with semantic explanations, showcasing the necessity of explanation computing in many fields to bridge the trust gap between humans and AI. Our research fosters trust in AI by making it more transparent, supports informed policy recommendation, enhances public safety, and ensures AI systems are reliable and accurate, contributing to a more knowledgeable and resilient society.


%\begingroup
%\setlength{\columnsep}{5pt}%
%\begin{figure}[!h]
\begin{wrapfigure}{r}[0pt]{0.5\textwidth}
\centering
\includegraphics[width=0.5\textwidth]{figures/graph.png}
\vspace{-.5cm}
\caption{Holistic graph embedding}
\vspace{-.5cm}
\end{wrapfigure}
%\end{figure}
%\endgroup

\textbf{3. Advances graph data management by developing superior unsupervised graph alignment and embedding techniques applied across various domains, including explainable AI, social media, and misinformation detection:} ICDE'23 (CORE $A^*$), two TKDE'23 (IF 9.235, CORE $A^*$), PR'22 (IF 8.518, CORE $A^*$), ICDE'22, EMNLP'22 (CORE $A^*$),  TKDE'21, ICDE'21, ICDE'20, VLDBJ'17 (CORE $A^*$), ICDE'14, SIGIR'13, etc.

My research has significantly advanced graph data management, developing an unsupervised graph alignment framework that outperforms supervised baselines without labels. I have also designed several streaming graph management methods, including vector-based indexing, cache management, and query streaming processing, and novel unsupervised graph embedding techniques for tasks like node classification, link prediction, subgraph isomorphism, and graph classification. Notably, I revolutionized subgraph isomorphisms for multiple graph queries under streaming settings, overcoming complex graph traversal challenges . My methods have been applied in explainable AI, social media, data filtering, and information retrieval. For instance, I developed a graph-based anomaly detection framework to detect misinformation like fake news and rumors, helping to restore public trust, demonstrating the efficacy of my graph-based approach in large-scale networks and systems. These advancements help create safer, more reliable digital environments and foster informed communities.
